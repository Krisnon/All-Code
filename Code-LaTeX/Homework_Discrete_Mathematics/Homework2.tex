\documentclass[UTF8]{ctexart}

\title{\heiti 第二次离散数学作业}
\author{\kaishu 闫嘉明}
\date{\today}

\begin{document}
    \maketitle
    \section{24}
    Let C(x) be the propositional function "x" is in your class.
    \subsection{a}
    Let M(x) be the propositional function "x" has a mobile phone.\\
    (1)$\forall x M(x)$\\
    (2)$\forall x (C(x) \wedge M(x))$
    \subsection{c}
    Let S(x) be the propositional function "x" can swim.\\
    (1)$\exists x \neg S(x)$\\
    (2)$\exists x (C(x) \wedge \neg S(x))$
    \section{37}
    \subsection{a}
    Let V(x) be the propositional function "x" qualifies as a vip.
    Let F(x,y) be the propositional function "x" flys more than "y" miles a year.
    Let N(x,y) be the propositional function "x" flys more than "y" times a year.\\
    $\forall x((F(x,25000) \vee N(x,25))\rightarrow V(x))$
    \subsection{b}
    Let M(x) be the propositional function "x" is male.
    Let T(x,y) be the propositional function "x" is able to finish a marathon in less than y hours.
    Let Q(x) be the propositional function "x" is qualified for the marathon.\\
    $\forall x (((M(x) \wedge T(x,3)) \vee (\neg M(x) \wedge T(x,3.5)))\rightarrow Q(x))$
    \subsection{c}
    Let Q(x) be the propositional function "x" is qualified for a master's degree.
    Let B(x) be the propositional function "x" gets higher than B in every subject.
    Let S(x,y) be the propositional function "x" gets more than "y" credits.
    Let P(x) be the propositional function "x" passed the master thesis defense.\\
    $\forall x ((B(x) \wedge (S(x,60) \vee (S(x,45) \wedge P(x))))\rightarrow Q(x))$
    \subsection{d}
    Let V(x) be the propositional function "x" has learned 21 credit course in a term.
    Let A(x) be the propositional function "x" got A for all course.
    Let S(x) be the propositional function "x" is a student.\\
    $\exists x (S(x)\wedge (V(x)\wedge A(x)))$
    \section{43}
    They are not equivalent. For example, if P(x) is sometimes true and sometimes false, then $\forall x (P(x)\rightarrow Q(x))$ is false but $\forall x P(x)\rightarrow \forall x Q(x)$ is true.
    \section{62}
    \subsection{a}
    $\forall x (P(x)\rightarrow \neg S(x))$
    \subsection{b}
    $\forall x (R(x)\rightarrow S(x))$
    \subsection{c}
    $\forall x (Q(x)\rightarrow P(x))$
    \subsection{d}
    $\forall x (Q(x)\rightarrow \neg R(x))$
    \subsection{e}
    Yes, we can. From (a) and (c), we know that $\forall x (Q(X)\rightarrow \neg S(x))$. But according to (b), $\forall x (R(x)\rightarrow S(x))$. So $\forall x (Q(x)\rightarrow \neg R(x))$.
\end{document}