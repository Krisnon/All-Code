%导言区
\documentclass[UTF8]{article}

\usepackage[UTF8]{ctex}

%\documentclass[UTF8]{article} & \usepackage[UTF8]{ctex}
%\documentclass[UTF8]{ctexart}
%这两者最终效果相同
%article, report, book 三者对应 ctexart, ctexep, ctexbook

\newcommand{\degree}{^\circ} %用 "\newcommand{}{}" 定义不存在的命令

\title{\heiti 我的第一份LaTeX文档}
\author{\kaishu 闫嘉明} %用前缀“\xxxxxx”来定义所使用的字体
\date{\heiti \today}

%正文区(文稿区)
\begin{document}
    \maketitle
    Hello \LaTeX

    %可以键入一个空行来区分段落,并且不论有几个空行,编译器均认为只有一行
    Let's input a formula $f(x)=3x^2+6x$. % "$$" 用于标识数学符号
    And now let it show in a separate line $$f(x)=g(x)$$ And here comes it.
    % "$$$$" 用于让数学公式单独显示在一行中

    The angle of C is: $$\angle C=90 \degree$$ %“\xxxxx” 命令与文本间要有空格来区分命令和正文
    勾股定理表述为:
    \begin{equation}
        AB^2=BC^2+AC^2
    \end{equation} 
    %"\begin{equation} & \end{equation}" 用于表述带尾部序列标志的方程
    
\end{document} %有且仅有一个document环境