\documentclass[UTF8]{ctexart}

\usepackage{amsmath}

\title{\heiti Matrix}
\author{\kaishu 闫嘉明} 
\date{\heiti \today}

\begin{document}
\maketitle
%matrix环境---无符号
%用数学公式标识符括起来
$
\begin{matrix}
    0 & 1 \\
    1 & 0
\end{matrix} \qquad
$
%pmatrix环境---()
\[
\begin{pmatrix}
    0 & -i \\
    i & 0
\end{pmatrix} \qquad
\]
%bmatrix环境---[]
\[
\begin{bmatrix}
    0 & 1 \\
    1 & 0
\end{bmatrix} \qquad
\]
%Bmatrix环境---{}
\[
\begin{Bmatrix}
    0 & 1 \\
    1 & 0
\end{Bmatrix} \qquad
\]
%vmatrix环境---||
\[
\begin{vmatrix}
    0 & 1 \\
    1 & 0
\end{vmatrix} \qquad
\]
%Vmatrix环境---||||
\[
\begin{Vmatrix}
    0 & 1 \\
    1 & 0
\end{Vmatrix}
\]
%矩阵常用省略号
$$
\begin{bmatrix}
    a_{11} & \dots & a{1n} \\
    & \ddots & \vdots \\
    0 & & a_{nn}
\end{bmatrix}_{n \times n}^n
$$
%可以为矩阵内元素以及矩阵本身加下标和上标
%分块矩阵(即矩阵嵌套)
$
\begin{pmatrix}
    \begin{matrix} 1 & 0\\0 & 1 \end{matrix} & \text{\Large 0}\\
    \text{\Large 0} & \begin{matrix} 1 & 0\\0 & 1 \end{matrix}
\end{pmatrix}
$
%\text{}用于临时切换至文本模式
%三角矩阵
$
\begin{pmatrix}
    a_{11} & a_{12} & \cdots & a_{1n}\\
    & a_{22} & \cdots &a_{2n}\\
    & & \ddots & \vdots\\
    \multicolumn{2}{c}{\raisebox{1.3ex}[0pt]{\Huge 0}} & & a_{nn}
\end{pmatrix}
$
%跨列省略号:\hdotsfor{<列数>}
$
\begin{vmatrix}
    1 & \frac 12 & \dots & \frac 1n \\
    \hdotsfor{4}\\
    m & \frac m2 & \dots & \frac mn
\end{vmatrix}
$

%行内小矩阵(smallmatrix)环境
复数$z=(x,y)$也可用矩阵
$
\left(
\begin{smallmatrix}
    x & y \\ y & x
\end{smallmatrix}
\right)
$
来表示

%array环境(类似于表格环境tabular)
$
\begin{array}{r|r|r}
    1 & 2 & 3\\
    \hline %用于生成跨所有列的横线
    4 & 5 & 6\\
    \hline
    7 & 8 & 9
\end{array}
$
\end{document}