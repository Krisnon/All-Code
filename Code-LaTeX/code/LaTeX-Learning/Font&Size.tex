\documentclass[UTF8]{ctexart}

\title{\heiti Font and Size}
\author{\kaishu 闫嘉明} 
\date{\heiti \today}

\begin{document}
\maketitle
%%%%%%%%%%%%%%%%    
%字体族设定(罗马字体、无衬线字体、打字机字体)
    \textrm{Roman Family}
    
    \textsf{Sans Serif Family}
    
    \texttt{Typewriter Family}
    %如上的声明可以定义{}内的文字字体族
    
    Font

    \sffamily Sans Serif Family in this line
    %如上的声明可以定义该声明之后所有文字的字体族

    and all the text behind
    
    {\rmfamily Roman Family}%带上{}可以使作用范围局限在{}内
    
    {\ttfamily Typewriter family}

    And back to Sans
    
    \ttfamily%出现新的命令后会覆盖之前命令的效果

    I'm Typewriter Family.
    \sffamily
    

%%%%%%%%%%%%
%字体系列设置(粗细、宽度)
    \textmd{Medium Series}  \textbf{Boldface Series}
    
    {\mdseries Medium Series}  {\bfseries Boldface series}

    \rmfamily
%%%%%%%%%%%
%字体形状(直立、斜体、伪斜体、小型大写)
    \textup{Upright Shape}  \textit{Italic Shape} 

    \textsl{Slanted Shape}  \textsc{Small Caps Shape}%Slanted shape 不适用于sans字体族

    {\upshape Upright Shape}  {\itshape Italic Shape}  
    
    {\slshape Slanted Shape}  {\scshape Small Caps Shape}


%%%%%%%%%%%
%中文字体
    {\songti 宋体}  {\heiti 黑体}  {\fangsong 仿宋}  {\kaishu 楷书}

    中文字体的\textbf{粗体}与\textit{斜体}


%%%%%%%%%%%
%字体大小
    {\tiny Hello}
    {\scriptsize Hello}
    {\footnotesize Hello}
    {\small Hello}
    {\normalsize Hello}
    {\large Hello}
    {\Large Hello}
    {\LARGE Hello}
    {\huge Hello}
    {\Huge Hello}

%中文字号设置命令
\zihao{0} 你好!
\end{document}