\documentclass[UTF8]{ctexart}
%ctexart中section默认居中
%使用article+ctex时section默认左对齐

\title{Section}
\author{\kaishu 闫嘉明} %用前缀“\xxxxxx”来定义所使用的字体
\date{\heiti \today}

\begin{document}
    \maketitle
    \section{引言}
    CTEX 宏集是面向中文排版的通用 LATEX 排版框架,为中文 LATEX 文档提供了汉字输出支持、
标点压缩、字体字号命令、标题文字汉化、中文版式调整、数字日期转换等支持功能,可适应论
文、报告、书籍、幻灯片等不同类型的中文文档。\\CTEX 宏集支持 LATEX、pdfLATEX、XƎLATEX、LuaLATEX、upLATEX 等多种不同的编译方式,并
为它们提供了统一的界面。 \par 主要功能由宏包 ctex 和中文文档类 ctexart、ctexrep、ctexbook 和
ctexbeamer 实现。
    %\\只能换行,并不形成新的段落
    %"\par" 为段落生成命令,不过一般用空行
    \section{实验方法}
    \section{实验结果}
    \subsection{数据}
    \subsection{图表}
    \subsubsection{实验条件}
    \subsection{结果分析}
    \section{结论}
    \section{致谢}
    %更多细节设置自行查阅ctex.pdf
\end{document}