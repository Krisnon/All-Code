\documentclass{article}

\usepackage{ctex}
\usepackage{amsmath}
%需要解决label的问题
\begin{document}
    \section{简介}
    \LaTeX{}将排版内容分成了文本模式和数学模式
    \section{行内公式}
    \subsection{美元符号}
    交换律是$a+b=b+a$.
    \subsection{小括号}
    交换律是\(a+b=b+a\).
    \subsection{math环境}
    交换律是 \begin{math}a+b=b+a\end{math}
    \section{上下标}
    \subsection{上标}
    $3x^2$
    
    $3x^{20}$%多于一位或使用多项式的上标时要用{}括起来
    \subsection{下标}
    $a_1, a_2, ......$

    $a_{200}, a_{3x_2+6}$%同理
    \section{希腊字母}
    $\alpha$
    $\beta$
    $\delta$
    $\gamma$
    $\Gamma$
    $\pi$
    $\Pi$%$$内即可调用其英文名构成的命令
    
    $\alpha^2=\beta^3$
    \section{数学函数}
    $\sin$
    $\log$
    $\ln$
    $\arcsin$%与希腊字母相同
    
    $\sin^2 x+\cos^2 x=1$
    
    $\sqrt{2}$
    $\sqrt{x^2+y^2}$
    $\sqrt[4]{x^3}$%[]用以限定开根次方
    \section{分式}
    1.$3/4$
    
    2.$\frac{3}{4}$%\frac会使分子分母依照上下排布

    3.$\frac{1}{1+ \frac{1}{x}}$
    \section{行间符号}
    \subsection{美元符号}
    比如说交换律是$$a+b=b+a$$这样的
    \subsection{中括号}
    比如说交换律是\[a+b=b+a\]这样的
    \subsection{displaymath环境}
    比如说交换律是
    \begin{displaymath}
    a+b=b+a
    \end{displaymath}
    这样的
    \subsection{自动编号公式equation环境}
    交换律见式 
    \begin{equation}
        a+b=b+a
    \end{equation}
    这样的
    
    \subsection{不编号公式equation*环境}
    比如说交换律是
    \begin{equation*}
    a+b=b+a
    \end{equation*}
    这样的
\end{document}